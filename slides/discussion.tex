%%%%%%%%%%%%%%%%%%%%%%%%%%%%%%%%%%%%%%%%%
% Beamer Presentation
% LaTeX Template
% Version 1.0 (10/11/12)
%
% This template has been downloaded from:
% http://www.LaTeXTemplates.com
%
% License:
% CC BY-NC-SA 3.0 (http://creativecommons.org/licenses/by-nc-sa/3.0/)
%
%%%%%%%%%%%%%%%%%%%%%%%%%%%%%%%%%%%%%%%%%

%----------------------------------------------------------------------------------------
%	PACKAGES AND THEMES
%----------------------------------------------------------------------------------------

\documentclass{beamer}

\mode<presentation> {

% The Beamer class comes with a number of default slide themes
% which change the colors and layouts of slides. Below this is a list
% of all the themes, uncomment each in turn to see what they look like.

%\usetheme{default}
%\usetheme{AnnArbor}
%\usetheme{Antibes}
%\usetheme{Bergen}
%\usetheme{Berkeley}
%\usetheme{Berlin}
%\usetheme{Boadilla}
%\usetheme{CambridgeUS}
%\usetheme{Copenhagen}
%\usetheme{Darmstadt}
%\usetheme{Dresden}
%\usetheme{Frankfurt}
%\usetheme{Goettingen}
%\usetheme{Hannover}
%\usetheme{Ilmenau}
%\usetheme{JuanLesPins}
%\usetheme{Luebeck}
\usetheme{Madrid}
%\usetheme{Malmoe}
%\usetheme{Marburg}
%\usetheme{Montpellier}
%\usetheme{PaloAlto}
%\usetheme{Pittsburgh}
%\usetheme{Rochester}
%\usetheme{Singapore}
%\usetheme{Szeged}
%\usetheme{Warsaw}

% As well as themes, the Beamer class has a number of color themes
% for any slide theme. Uncomment each of these in turn to see how it
% changes the colors of your current slide theme.

%\usecolortheme{albatross}
%\usecolortheme{beaver}
%\usecolortheme{beetle}
%\usecolortheme{crane}
%\usecolortheme{dolphin}
%\usecolortheme{dove}
%\usecolortheme{fly}
%\usecolortheme{lily}
%\usecolortheme{orchid}
%\usecolortheme{rose}
%\usecolortheme{seagull}
%\usecolortheme{seahorse}
%\usecolortheme{whale}
%\usecolortheme{wolverine}

%\setbeamertemplate{footline} % To remove the footer line in all slides uncomment this line
%\setbeamertemplate{footline}[page number] % To replace the footer line in all slides with a simple slide count uncomment this line

%\setbeamertemplate{navigation symbols}{} % To remove the navigation symbols from the bottom of all slides uncomment this line
}

\usepackage{graphicx} % Allows including images
\usepackage{booktabs} % Allows the use of \toprule, \midrule and \bottomrule in tables

%----------------------------------------------------------------------------------------
%	TITLE PAGE
%----------------------------------------------------------------------------------------

\title[]{Discussion of First Year Paper} % The short title appears at the bottom of every slide, the full title is only on the title page

\author{Patrick Shultz} % Your name

\date{\today} % Date, can be changed to a custom date

\begin{document}

%\begin{frame}
%\titlepage % Print the title page as the first slide
%\end{frame}

%----------------------------------------------------------------------------------------
%	PRESENTATION SLIDES
%----------------------------------------------------------------------------------------

%------------------------------------------------
\section{First Section} % Sections can be created in order to organize your presentation into discrete blocks, all sections and subsections are automatically printed in the table of contents as an overview of the talk
%------------------------------------------------

\begin{frame}
\frametitle{General Idea}
\begin{itemize}
\item What is the role of uncertainty in the transmission of monetary policy to financial markets?



\item I use options on Eurudollar futures contracts that allow me to investigate the effect of uncertainty from 2011-today. 

\end{itemize}
\end{frame}

%------------------------------------------------

\begin{frame}
\frametitle{Data}
\begin{itemize}
	\item Quarterly Eurodollar futures contracts with expirations every March, June, September, and December and the corresponding puts and calls.
	\item Options on Eurodollar futures are among the most actively traded Exchange-listed Interest Rate options in the world. 
	\item The payoffs of these contracts are tied to the three-month LIBOR rate. 
    

\end{itemize}

\end{frame}

%------------------------------------------------

\begin{frame}
\frametitle{Uncertainty measure}
\begin{itemize}
	\item The risk-neutral variance of the level of LIBOR is pinned down by out of the money puts and calls on Eurodollar futures.  
	\item Bauer, Lakdawala, and Mueller (2019) claim:
	\begin{equation}
	\begin{split}
			Var_{t}(L_{T}) &= Var_{t}(F_{T, T}) = E_{t}F_{T, T}^{2} - (E_{t}F_{T, T})^{2} = E_{t}F_{T, T}^{2} - F_{t, T}^{2}\\
			&= \dfrac{2}{P_{t, T}}\left(\int_{0}^{F_{t, T}}p(K)dK + \int_{F_{t, T}}^{\infty} c(K) dK\right)\\
	\end{split}
	\end{equation}
	where $L_{t}$ is the LIBOR, $F_{t, T}$ is the price of the futures contract, $P_{t, T}$ is the zero coupon bond price, and $c(K)$ and $p(K)$ are the time-$t$ prices of calls and puts with strike $K$. 
	\item Empirically, this essentially comes down to estimating the terms in the integrals each trading day. 
\end{itemize}
\end{frame}

%------------------------------------------------
\begin{frame}
\frametitle{Daily fit of options prices}
\begin{itemize}
	\item First, convert prices into implied volatilities using Black-Scholes. Second, fit a cubic spline to the implied volatilities. Third, map the volatility curve back into price space. 
	\begin{figure}
		\centering
		\includegraphics[scale=0.4]{example_of_spline.png}
	\end{figure}
\end{itemize}
\end{frame}

%------------------------------------------------

\begin{frame}
\frametitle{What is this measuring?}
	\begin{itemize}
		\item Three main components:
		\begin{enumerate}
			\item Uncertainty around future monetary policy. 
			\item Uncertainty around the LIBOR-OIS spread
			\item A variance risk premium, since the construction of uncertainty is done under the risk neutral measure. 
		\end{enumerate}
	\item Changes in the LIBOR-OIS spread tend to be small (less than a basis point on average) on FOMC days, so movements in this measure around announcements are likely to be due to changes in uncertainty regarding monetary policy and the variance risk premium. 
	\end{itemize}
\end{frame}

%------------------------------------------------
\begin{frame}
\frametitle{Eurodollar based uncertainty (EDU) at Fixed Horizons}
\begin{figure}
	\centering
	\includegraphics[scale=0.35]{mpu.png}
	\caption{Risk-neutral standard deviation of three-month LIBOR rate at horizons of 0.5, 1, 1.5, 2, 2.5 years, estimated from Eurodollar futures and options. Sample period: 2011-10-18 to 2019-07-19.}
\end{figure}

\end{frame}

\begin{frame}
\frametitle{Response of EDU to FOMC announcements}
\begin{figure}
	\centering
	\includegraphics[scale=0.35]{../figures/mpu_diff_fomc.png}
	\caption{Response to FOMC announcements. Vertical line starts at day prior to FOMC announcement}
\end{figure}
\end{frame}


\begin{frame}
\frametitle{Response of yields to EDU on FOMC days}
\begin{figure}
	\centering
	\includegraphics[scale=0.35]{../figures/yield_responses_US_fomc_control_v_no_control.png}
	\caption{$\hat{\beta_{1}}$ from regression of the form $\Delta y_{t}^{(n)} = \alpha + \beta_{1} \Delta EDU_{t} + \beta_{2} \Delta EDS$. The left panel omits $\Delta EDS$ as a control. }
\end{figure}

\end{frame}

\end{document} 