\title{\textbf{Interest Rate Uncertainty, Monetary Policy, and Asset Prices}}
\author{
        Patrick Shultz\\
        The Wharton School at the University of Pennsylvania
}
\date{\today}

\documentclass[11pt]{article}
\usepackage{amsmath}
\usepackage{amsfonts}
\usepackage{xcolor}
\usepackage{bbm}
\usepackage{subcaption}
\usepackage{graphicx} % Allows including images
\usepackage[margin=1in]{geometry}
\usepackage{setspace}
\begin{document}
\maketitle

\begin{abstract}
Despite the federal funds rate being fixed at the zero lower bound from 2008-2015, uncertainty regarding the path of short term interest rates persisted in this time period. Despite low variance in the Federal Funds rate, asset prices continued to have significant movements on FOMC announcements day. This paper investigates what about FOMC announcement, if not changes in interest rates, drives asset prices on these day. To this end, I develop derivatives based measures of uncertainty, the probability of interest rates being at the zero lower bound at specified maturities in the future, and the risk of interest rate rising over the maturity of the underlying options. I find that surprises in these measures have quantitatively different effects on asset prices than surprise changes in the level and expected path of interest rates. 

\textbf{The Null Hypothesis of the paper is that only the first moment of the policy variable matters for asset prices. }


 
\end{abstract}

\section{Introduction}
\doublespacing

Even though the federal funds rate was held fixed between 0 and 25 basis points from December 16, 2008 to December 16, 2015, uncertainty regarding the path of interest rates persisted. Furthermore changes in the level of interest rates over this period tended to be very small {\color{red} Add simply calculation of targets changes in the three-month London Interbank Offered Rate (LIBOR/Fed Funds Rate}. Given the relatively small changes in the Federal Funds Rate over this time period, it is natural to wonder why asset prices seemed to be responsive to policy announcement despite almost no surprises in the key policy variable or the path of the variable on announcement days\footnote{Create MP surpise measures from federal funds rate. Then explain that firms don't actually get to borrow at FFR so LIBOR is the logical rate to use. Or just use ED surprises.}. This paper seeks to answer that question by creating three new measures of monetary policy and investigating the relationship between these measures, the monetary policy announcements, and asset prices. {\color{red} Look at the effect of change in just eurodollar futures over entire sample and the effect on asset prices, then check if there is a puzzle of asset prices being more sensitive to asset prices at the ZLB than in the earlier part of the sample.}

To this end, I develop three measure that capture dimensions of monetary policy aside the first moment of policy (i.e. the level of the short term interest rate) where the literature has generally focused. Specifically, I measure the risk-neutral variance of the level of short term rates, the risk-neutral probability of the policy rate being at the ZLB in $\tau$ years, and the risk-neutral probability of a large interest rate hike. 

After developing these measure I follow Rigobon and Sack (2002), who identify a significant response of the stock market to interest rate surprises derived from eurodollar futures. Rather than using their measure of interest rate surprises, I focus on surprises to the non-mean surprises in the distribution of interest rate expectations. 

The relationship between these non-first moment surprises and asset prices is an important topic for several reasons. The recent financial crisis showed that existing policy frameworks can be substantially constrained in times of crisis. Hence, having alternative tools to effect financial markets could be important during future recessions.  Much of the transmission of monetary policy comes through the effect short term interest rates have on other assets, such as long term bonds, which in turn influence borrowing costs and real economic activity. However, relatively little research has been done on the effect of uncertainty around the short term rate and its effect on investment. Theoretical research (Abel 199X) showed that this should have an important effectt.   


\textbf{Insert literature review/introduction here}

Overall, this paper is an extension on the previous literature which investigates the effect of monetary policy surprises on asset prices with the twist that it looks at non-mean measures of short term interest rates. 
\section{Institutional Details of the Eurodollar Market}
Eurodollars are simply time deposits denominated in United States dollars at banks outside of the US. The markets initially referred specifically to dollar deposits in Europe, but has since expanded to cover the aforementioned definition. It is important to note that there is no connection to a Eurodollar deposit and the euro-dollar exchange rate. Eurodollar futures contracts are derivatives on interest rates paid on these deposits and are traded at the Chicago Mercantile Exchange (CME). They are primarily used by corporations and banks to lock in an interest rate today for money they need to borrow or loan in the future. The final settlement price is determined by the three-month London Interbank Offered Rate (LIBOR) on the last trading day. Hence, we can consider the futures implied interest rate as a market based expectation of future interest rates. Eurodollar options are American and the payoffs are dependent on the final price of the underlying Eurodollar future. Together, Eurodollar futures and options together can be used to hedge interest rate fluctuations of dollar deposits, and hence can provide important information of market particpants' expectations of future US monetary policy. 

Estimates of market-based expected value of future short term interest rates can easily be inferred from futures contracts of various maturities. However, since I am interested in market based measures of parts of the distribution that are not the mean, I must turn to Eurodollar options prices. 

The advantages of using the Eurodollar market rather than the Federal Funds market are threefold.  First, the Eurodollar market is more liquid than the Federal Funds market. In fact, as shown in Figure \ref{fig:liquidity}, the only market with higher annual volume as of 2018 was the foreign exchange market. eurodollar futures and options' average daily trading volume exceeds 4.14 million contracts (\$4 trillion notional), and open interest is over 50.21 million contracts (\$50 trillion notional) \footnote{see:https://www.cmegroup.com/education/articles-and-reports/whats-next-for-libor-eurodollar-futures.html }.

Second,  \$160 trillion of financial obligations are based on USD LIBOR, including syndicated loans, home mortgages, student loans, retail bank deposits, structured products, OTC interest rate swaps, forward rate agreements, and OTC interbank cross-currency swaps. Hence, we are interested in the effect of FOMC announcements on the interest rate these products are tied to. 

Third, overnight indexed swaps (OIS) require collateralization, which removes counterparty risk, and because of the costs associated with posting and maintaining collateral, the discount rate on these products requires an adjustment (Collin-Defresne and Solnik (2001), Johannes and Sundaresan (2007), and Bibkov and Chernov (2011)). 

The sample runs from 2011-2019, which may seem short, but still includes a total of 65 FOMC policy meetings. Previous studies have used similarly short samples. For example, Rigobon and Sack  rely on data from 1994 to 2001 - a period over which a total of 78 policy meetings took place. 

\subsection{Eurodollar Futures Contracts}
{\color{blue}Eurodollar futures contracts are written directly on the 90 day LIBOR, which I denote as $L_{t}$, as opposed to a traded asset. The Eurodollar futures rate with maturity $\tau$ is quoted as 
\begin{equation*}
	F_{t}(\tau) = 100 - f_{t}(\tau)
\end{equation*}
where $f_{t}(\tau)$ is future LIBOR rate. The contract settles in cash based on $100 - L_{t + \tau}(d)$. Eurodollar futures and associated options contracts are issued at a quarterly frequency. See Bikbov and Chernov 2011.} The contract gives a way for market participants to lock in interest rates for a given horizon where buying the contract is equivalent to loaning money and purchasing a contract is equivalent to borrowing at the implied rate on the contract. Hence, these contracts give a market based expectation of the level of LIBOR in $\tau$ years. Since Eurodollars are U.S. dollar-denominated deposits held in banks outside of the United States, the price of the futures contract reflects the market's expectations of the 3-month U.S. dollar LIBOR interest rate anticipated on the settlement date of the contract. \footnote{Example: If an investor buys one Eurodollar futures contract at \%96.00 and the price rises to \$96.02, this corresponds to an implied settlement of LIBOR at \%3.98. The buyer of the contract will have made \$50 (two basis points $\times$ 25)}.\footnote{example: let's assume that on Sept. 1, the December eurodollar futures contract price was 96.00, implying an interest rate of \%4.0, and that at expiry in December the final closing price is \$95.00, reflecting an interest rate of \%5.00. Shorting Eurodollar contracts would hedge this increase. Basically, the futures contracts just give the right to earn a rate on a deposit over a certain period of time. }. The final settlement of an expiring Eurodollar futures contract is determined by reference to the three-month LIBOR on the last trading day. Thus, movements in the Eurodollar futures market provide insight as to where market participants think LIBOR will be in the future. 

I use these securities as a way to predict US monetary policy. Since these contracts are directly tied to LIBOR rather than to the federal funds rate, the quality of these forecasts is dependent on how well LIBOR tracks the federal funds rate. As shown in SECTION ??, LIBOR tracks the federal funds rate closely and hence is a good proxy for US monetary policy. Additionally, Gurkanyak, Sack, and Swanson (Market Based Measures 2007) show that for horizons past six months, the focus of this paper, the term federal funds, term eurodollars, and eurodollar futures markets all seem to forecast future US monetary policy equally well.  \\
{\color{red}\textbf{Add figure of daily volume over sample}}\\



\subsection{What are Eurodollar options? }
 Options on Eurodollar futures are among the most actively traded in the world and have recently had a daily volme of FIXME X times the federal funds options market. Additionally, Eurodollar futures and options are available at longer maturities, making them more useful for this exercise. Due to the payoff structure a call on the Eurodollar futures contract should be thought of as a put on LIBOR at time $T$ and puts on the futures contract should be thought of as a call on LIBOR. \\


 
%\section{LIBOR-OIS Spread}
%\textbf{LIBOR:} Represents the daily average rate that banks can charge one another in order to borrow cash overnight. It is an unsecured loan reflecting the creditworthiness of the bank borrowers. The world's largest lenders on the London Interbank Market create a daily average of their cost of funds, which is then reported as LIBOR. \\
%
%\noindent \textbf{The Overnight Interest Swap:} This rate represents the rate where high quality borrowers can swap a variable rate payment for a fixed rate or vice versa. The monthly payment settlements only represent the difference in the cash flows being exchanged, so there is no principal at risk. Credit risk is not a factor in determining the OIS rate. \\
%
%If LIBOR increases relative to the OIS rate, it means it is becoming more expensive for banks to borrow money from one another when compared to a near ''risk free" rate paid by high quality swappers in the OIS market.
%
%LIBOR OIS spread vs. MPU measure 
%\begin{figure}
%	\centering
%	\includegraphics[scale=0.5]{../figures/libor_ois_scatter.png}
%	\caption{LIBOR-OIS Spread vs EDU measure. $\beta = -0.06, |t| = 0.554$}
%	\label{fig:spread_mpu}
%\end{figure}


LIBOR is an important rate to study for several reasons: it is the rate benchmark for \$200 trillion of dollar-denominated financial products, corporate bank lending floating rate loans have coupons indexed mostly to LIBOR, Eurodollar options are substantially more liquid than options on federal funds rate futures and are among the most liquid interest rate options in the world, and the LIBOR-OIS spread is assumed to be a measure of the health of banks because it reflects what banks believe is the risk of default associatd with lending to other banks\footnote{see: https://files.stlouisfed.org/files/htdocs/publications/es/09/ES0924.pdf}.
[5] LIBOR is the main reference rate in U.S. economy and as of 2016 outstanding business loans indexed to USD LIBOR amounted to 3.4 trillion, and for syndicated loans, amounting to 1.5 trillion and over 90\% of the overall volume.  (Jermann 2019), so it is natural to ask what is the affect of FOMC announcements on uncertainty around this rate. Hence, if we want a measure of interest rate uncertainty for the U.S. economy LIBOR is a more appropriate rate than the federal funds rate.  \\

\section{Risk Neutral Variance}
{\color{blue}I temporarily assume that prices of puts and calls are observed expiring at time $T$ are perfectly observable at all strikes $K$. Additionally, I use asterisks to denote quantities calculated with risk-neutral probabilities and $M_{T}$ to denote the stochastic discount factor (SDF) that prices time $T$ payoffs from the perspective of time $t$. We can price any payoff $X_{T}$ with either the SDF or by computing the risk-neutral probabilities and discounting at the gross riskless rate, $R_{f, t}$. In the SDF notation, we have 
	\begin{equation*}
	P_{t} = E\left[M_{t}X_{T}\right]
	\end{equation*}
	and in risk neutral notation, we have 
	\begin{equation*}
	P_{t} = \dfrac{1}{R_{f,t}}E^{*}X_{T}
	\end{equation*}} 

Figure \ref{fig:spline} plots a collection of time-$t$ prices of puts and calls expiring at $T$ with strikes $K$. The figure illustrates two useful facts regarding options and forward prices. 
The first is that put and call prices are convex functions of price, which allows us to deal with the issue that options prices are only observed discretely, not continuously. The second is that the forward price of the underlying asset satisfies

\begin{equation}
F_{t, T} = E_{t}^{*}\left[L_{T}\right]
\label{eq:forward_martingale}
\end{equation}
that is, the implied futures rate is equal to the expected value of the future LIBOR rate under the risk-neutral measure.

Using Equation \ref{eq:forward_martingale}, the conditional risk-neutral variance of the \textit{level} of LIBOR can be given as
\begin{equation}
\begin{split}
Var_{t}^{*}(L_{T}) = Var_{t}^{*}(F_{T, T})= E_{t}^{*}F_{T, T}^{2}-(E_{t}^{*}F_{T, T})^{2}=E_{t}^{*}F_{T, T}^{2} - F_{t, T}^{2}
\end{split}
\label{eq:LIBOR_var}
\end{equation}
where $L_{t}$ is the LIBOR, $F_{t, T}$ is the forward LIBOR rate, and $L_{T} = F_{T, T}$. The first equality holds by arbitrage, the second equality holds by definition of variances, and the final equality holds because the forward rate is a martingale under the $T$-forward measure. We can directly observe the second term of Equation \ref{eq:LIBOR_var}, so the challenge comes from calculating $\mathbb{E}_{t}^{*}F_{T, T}^{2}$-- the time $t$ price of the 'squared contract' that is paid at time $T$. 

How can we price this contract? Following the exposition of Martin (2018), we can consider the following portfolio. Suppose the current underlying price is $\$95.00$. We can buy two call options with a strike of $K = 95.5$; two calls with a strike of $96.00$; two calls with a strike of $96.00$ and so on. The payoff to such a strategy is indicated in Figure \ref{fig:squared_contract}. The solid and dotted lines overlap perfectly at integer values of $S_{T}$. Hence, the payoff of the squared contract is approximated by 
\begin{equation*}
\dfrac{1}{R_{f, t}}\mathbb{E}^{*}S_{T}^{2} \approx 2 \sum_{K=0.5, 1.5, ...}\text{call}_{t, T}(K)
\end{equation*}
It turns out that the squared contract can be priced exactly by replacing the sum with an integral
\begin{equation*}
\dfrac{1}{R_{f, t}}\mathbb{E}^{*}S_{T}^{2} = 2 \int_{K=0}^{\infty}\text{call}_{t, T}(K)dK
\end{equation*}
For our futures contracts this implies
\begin{equation}
%\dfrac{1}{R_{f, t}}\mathbb{E}^{*}_{t}F_{T, T}^{2}  = 2 \int_{0}^{\infty}\text{call}_{t, T}(K) dK
\label{eq:option_identity}
\end{equation}
where $c(K) = P_{t, T}E_{t}\max(0, F_{T, T}-K)$ is the time-$t$ price of a call option with strike $K$. 
Plugging Equation \ref{eq:option_identity} into Equation \ref{eq:LIBOR_var} yields 
\begin{equation}
\begin{split}
Var_{t}(L_{T}) &= \dfrac{2}{P_{t, T}}\left(\int_{0}^{F_{t, T}}p(K)dK + \int_{F_{t, T}}^{\infty}c(K)dK\right)\\
&=2 \int_{0}^{\infty}\left[\dfrac{c(K)}{P_{t, T}}- \max(0, F_{t, T}-K)\right]dK
\end{split}
\label{eq:oom_expression}
\end{equation}
The first equality shows that the risk-neutral variance can be written as a function of out-of-the-money puts and calls. Importantly, this volatility is a measure of uncertainty around the level of short term interest rates, not on the return of a Eurodollar futures contract.\footnote{Compare this to just straight using the VIX formula but applied to interest rate options.} \\

While this measure clearly must reflect uncertainty around monetary policy, it must also contain information about risk and risk aversion. Hence, it is not clear if increases in this measure are due to uncertainty around future policy rates or increasing risk aversion to unexpected changes in policy. 
The risk-neutral measure of $Var_{t}(L_{T})$ has three components:
\begin{itemize}
	\item First, it captures uncertainty around short term lending rates. 
	\item Second, it reflects uncertainty around credit risk due to the underlying being LIBOR rather than OIS. Sepcifically, it captures uncertainty regarding the LIBOR-OIS spread, which is a common benchmark for financial health.  
	\item Third, the measure captures a variance risk-premium in addition to ``real-world" variance, since it is a ``risk-adjusted" measurement of variance. 
\end{itemize}
While I cannot disentangle which component of this measure the Federal Reserve affects with monetary policy announcements, each component matters for financial decision makers and hence is important to study.  Using daily data, I investigate the behavior of this measure around 65 FOMC announcements.  


\textbf{Is this just correlated with current volatility?}
\textit{Measuring }
\subsection{Empirical Implementation}


\section{The Impact of Monetary Policy on Asset Prices}
This section demonstrates how to extract a measure of unanticipated changes in the short term interest rate and documents asset prices react to changes in monetary policy. Expectations of Fed policy actions are not directly observable, but futures markets on short term interest rates can provide a market based proxy for those expectations. \textbf{INSERT COMPARISON OF FED FUDNS IMPLIED PATH VS. EURODOLLAR FUTURES IMPLIED PATH HERE. } By relying on market based measures there is no issue of selecting a model from which a surprise would be derived. However, due to data availability, my analysis is limited to 1989 for Eurodollar futures and to the post 2011 period for Eurodollar options.

The question of interest is how to measure the unexpected change in the policy variable on date $t$, relative to the forecast made on date $t-1$. Denoting $x$ as the monetary policy variable of interest\footnote{this could be the level surprise, the change in uncertainty, or the change in the probabilty of hitting the zero lower bound.}, we want to measure 
\begin{equation*}
	x_{t} - E_{t-1}x_{t}
\end{equation*}

\section{Pricing Zero Lower Bound Risk}
DUring the 2007-2009 recession, the US economy experienced a sharp contraction and slow recovery. Simultaneously, the federal funds rate fell to zero and conventions monetary policy was constrained from 2008-2015 by the zero lower bound. Hence, the risk of reaching the ZLB proved to be of practical concern and is an issue that continues to concern policy makers. Given this heightened interest in the ZLB, it would be useful to consider a financial market based estimate of rearching the lower bound on interest rate. To consider this problem, I calculate price of a digital put option, that is the price of a claim to \$1 paid if and only if $S_{T} < \alpha S_{t}$. Since the prices of Eurodollar futures contracts are inversely related to the expected interest rate at expiration, we can think of a decline in price as an increase in expected interest rates. We can think of this as the amount market participants are willing to pay to hedge a $1-\alpha$ increase in interest rates. By pricing digital puts, I am able to construct the risk-neutral probability of short term interest rates increasing by $\alpha$ percent.  


\begin{equation} 
dput_{t, T}(K) = lim_{\epsilon \rightarrow 0} \frac{put_{t, T}(K+\epsilon)-put_{t, T}(K)}{\epsilon} = put_{t, T}^{'}(K)
\label{eq:putprime}
\end{equation}
Furthermore, we can derive the risk-neutral probability of this event by 
\begin{equation}
dput_{t, T}(K) =  \dfrac{1}{R_{f, t\to T}} \mathbb{E}_{t}^{*}\textbf{1}_{S_{T} < \alpha S_{t}} = \dfrac{1}{R_{f, t\to T}}\mathbb{P}^{*}(S_{T}<\alpha S_{t})
\label{eq:digital_put_price}
\end{equation}
where $\alpha < 1$.  Combining Equations \ref{eq:putprime} and \ref{eq:digital_put_price}, we get that the risk neutral probability of the Eurodollar futures contract declining by $(1-\alpha) \%$ is given by 
\begin{equation}
\mathbb{P}^{*}(S_{T}<\alpha S_{t}) = R_{f, t\to T}put_{t, T}'(K)
\label{eq:rate_hike_prob}
\end{equation}
Equation \ref{eq:rate_hike_prob} directly corresponds to the probability of interest ending up at a level higher than is currently implied by the underlying futures contract. 

{\color{pink} Can I change from the underlying to the current level of libor to get the probability of an interest rate increase. }




Next I consider the price of a digital call option, that is the price of a claim to \$1 paid if and only if $S_{T} > \beta S_{t}$, where $\beta > 1$. Pricing this instrument can be thought of as pricing a pixel with width $\beta S_{t} - S_{t}$, since LIBOR is truncated at the zero lower bound. Hence, we rely on the results of Breeden and Litzenberger (1978), which builds on Ross (1976) to show how to compute a risk-neutral expectation of the form
\begin{equation*}
\mathbb{E}^{*} g(S_{T})
\end{equation*}
for any random variable $S_{T}$ on which European options are traded and for any arbitrary function at time $T$.  Since a price of $\$100$ corresponds to an interest rate of $0\%$, we are interested in calculating the following risk-neutral distribution of $S_{T}$ as 
\begin{equation}
\mathbb{P}^{*}(S_{T} \in \left[S_{t}, \beta S_{t} \right]) = \mathbb{E}^{*}\textbf{1}_{S_{T}\in \left[S_{t}, \beta S_{t}\right]}
\label{eq:risk_neutral_pixel}
\end{equation}
which corresponds to using $g(x) = \textbf{1}_{S_{T}\in \left[S_{t}, \beta S_{t}\right]}$. Following Martin (2018), I call this instrument a pixel. The price of this pixel is given by 

\begin{equation}
\text{pixel call}_{t, T}(K) = \dfrac{1}{R_{f, t\to T}} \mathbb{E}^{*}\textbf{1}_{S_{T}\in \left[S_{t}, \beta S_{t}\right]}
\label{eq:pixel_price}
\end{equation}
By combining Equations \ref{eq:risk_neutral_pixel} and \ref{eq:pixel_price}, we see that the risk-neutral probability of the futures contract expiring at a price between $S_{t}$ and $\beta S_{t}$ is given by 
\begin{equation*}
\mathbb{P}^{*}(S_{T} \in \left[S_{t}, \beta S_{t} \right]) = R_{f, t\to T} \times \text{pixel call}_{t, T}(K)
\end{equation*}
I consider a $\beta \approx 1$ as the most interesting case since this roughly corresponds to the probability of remaining or returning to the zero lower bound. CALCULATE SPECIFIC BETA BY USING SUMMARY STAT OF LIBOR DURING ZLB PERIOD. \\

We can construct a strategy using calls to construct a payoff of $1$ if $S_{T} \in \left[S_{t}, \beta S_{t}\right]$ and $0$ otherwise as follows: buy $\frac{1}{\epsilon}$ calls with strike $S_{t} - \epsilon$, sell $\frac{1}{\epsilon}$ calls with strike $S_{t}$ and $\frac{1}{\epsilon}$ calls with strike $\beta S_{t}$, and buy $\frac{1}{\epsilon}$ calls with strike $\beta S_{t} + \epsilon$. By taking the limit of this portfolio with $\epsilon \to 0$, we construct a pixel. Specifically, we get that the price of the pixel is given by 
\begin{equation*}
\begin{split}
\lim\limits_{\epsilon \to 0} \dfrac{\text{call}_{t, T}(S_{t} - \epsilon) - \text{call}_{t, T}(S_{t})}{\epsilon} - \dfrac{\text{call}_{t, T}(\beta S_{t}) - \text{call}_{t, T}(\beta S_{t} + \epsilon)}{\epsilon} &= \text{call}^{'}_{t, T}(\beta S_{t}) - \text{call}^{'}_{t, T}(S_{t}) \\
&\approx \text{call}^{''}(S_{t})(\beta S_{t} - S_{t})
\end{split}
\end{equation*}
Hence, we see that the price of a claim to a call options that pays off if the underlying expires at least a proportion $\beta$ higher than the current price is dependent on the second derivative of the call price as a function of strike. The risk neutral probability is then given by 
\begin{equation*}
	\mathbb{P}^{*}(S_{T}\in\left[S_{t}, \beta S_{t}\right]) = R_{f, t\to T}( \text{call}^{'}_{t, T}(\beta S_{t}) - \text{call}^{'}_{t, T}(S_{t}) )
\end{equation*}


I now use this methodology to calculate the probability of short term interest rates being constrained by the zero lower bound. 





{\color{red} Next I consider whether or not these probabilities successfully predict changes in the price of the underlying asset. Specifically, I test the null hypothesis that $\beta = 0$ in the following regression. 
	\begin{equation*}
	\Delta LIBOR_{t, T} = \alpha + \beta rnpd_{t} + \epsilon_{t}
	\end{equation*}}


\section{FOMC announcements}

Compare this to level shocks and check if I can make some sort of claim of two distinct channels of monetary policy.

Regressions on credit spread, term premia, yields, exchange rate, and SP500.  

\section{Implied Variance vs. Realized Variance of LIBOR}


\textbf{CAN I FIND TWO DAYS WITH SIMILAR IMPLIED PATH BY EURODOLLAR FUTURES BUT DIFFERENT LEVELS OF UNCERTAINTY AROUND THAT PATH}

\textbf{Analyst prediction vs predication by market?}
\section{Conclusions}\label{conclusions}
I worked hard, and achieved very little.

\section{Appendix Figures }
\begin{figure}[!htb]
	\centering
	\includegraphics[scale=0.5]{../figures/volumes.png}
	\caption{Total volume of largest financial markets in 2018}
	\label{fig:liquidity}
\end{figure}
\begin{figure}
	\centering
	\includegraphics[scale=0.5]{../figures/libor_fed_funds.png}
	\caption{Libor and the effective federal funds rate, $\rho = 0.99$}
	\label{fig:libor_fed_funds_rate}
\end{figure}
\begin{figure}
	\centering
	\begin{subfigure}{.5\textwidth}
		\includegraphics[width=1\linewidth]{../figures/digital_puts.png}
		\caption{Probability of an $\alpha$ percent interest hike implied by options on Eurodollar futures contracts}
		\label{fig:hike_probs}
	\end{subfigure}%
	\begin{subfigure}{.5\textwidth}
		\includegraphics[width=1\linewidth]{../figures/zlb_prob.png}
		\caption{Probability changes around FOMC announcements. Vertical line represents day prior to announcement. }
		\label{fig:hike_probs_fomc}
	\end{subfigure}
\end{figure}
\begin{figure}
	\centering
	\includegraphics[scale=0.5]{../figures/example_of_spline.png}
	\caption{Time $t$ prices of puts and calls expiring at $T$}
	\label{fig:spline}
\end{figure}

\begin{figure}
	\centering
	\includegraphics[scale=0.5]{../figures/squared_contract.png}
	\caption{The payoff of $F_{T}^{2}$ (solid line) and the payoff to a portfolio consisted of two calls with strike $K = 0.5$, two calls with strike $K = 1.5$, two calls with strike $K = 2.5$, and so on. Individual option payoffs are indicated with dotted lines. }
	\label{fig:squared_contract}
\end{figure}

\begin{enumerate}
	\item Select out-of-the-money put and call contracts and collect closing prices
	\item Calculate the risk free interest rate and $P_{t, T}$ based on the zero-coupon yield curve of GSW. 
	\item Translate observed option prices into IVs, fit a cubic spline to IVs, and map IVs back to prices to obtain a smooth estimate of $c(K)$.
	\item Numerically calculate Equation \ref{eq:oom_expression} using the trapezoidal rule over a grid of 60 strikes. The square root is then an estimate of the standard of the LIBOR rate at the expiration of the contract.
\end{enumerate}
\begin{figure}
	\centering
	\includegraphics[scale=0.6]{../figures/mpu.png}
	\caption{Estimates of MPU for horizons of 0.5, 1, 1.5, 2, 2.5 years}
	\label{fig:mpu}
\end{figure}

\begin{figure}
	\centering
	\includegraphics[scale=0.6]{../figures/fomc_days_change.png}
	\caption{Percent change in LIBOR uncertainty on FOMC announcement days}
	\label{fig:fomc_days}
\end{figure}
\begin{figure}
	\centering
	\includegraphics[scale=0.6]{../figures/mpu_diff_fomc.png}
	\caption{Percent change in LIBOR uncertainty relative to day prior to FOMC announcements}
	\label{fig:fomc_cycle_chg}
\end{figure}


\begin{figure}
	\centering
	\includegraphics[scale=0.6]{../figures/term_premium_responses_fomc.png}
	\caption{Coefficients and confidence intervals for regression of $\Delta TP = \alpha + \beta \Delta mpu$ only on FOMC announcement days}
	\label{fig:term_premium_responses_fomc}
\end{figure}
\end{document}
This is never printed